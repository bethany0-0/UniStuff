

\documentclass{article}
 
\begin{document}

f(x)=y


\section{Describe a suitable chromosome representation of an individual}

Representation

As the value of x is not discrete, the most suitable representation will be a real-valued (continuous) representation. 

Due to the Hamming Cliffs in binary coding, the best representation for this exercise will be Gray coding. This is because, during mutations, we want to produce offspring similar to the parents (the fittest of the current population). Gray coding, changing a single bit causes a small change in the value of the real number it represents. Binary can up to double the real value with a single mutatoion which, though possibly advantageous during the initial search stages, may cause massive variations while attempting to find local minima.

Recombination

/////////////////////////Maybe?????????????
Not use intermediate recombination as this changes the genetic values. While the typical probem with this is that it can cause a value which is not valid, by the look the the unknown funciton, x is unbounded. The problem in this case will be that multiple return value are acceptable. This means that I can find multiple minimum which are all correct, so I do not want to find the average of two fit parents, I want the searched to remain independant to fidn each local minimum. 

For this reason I will use Discreate Recombination, specifically quadratic recombination mean. 

\[ x_j^' = \frac{1}{2} . \frac{(x_2j^2 - x_3j^2)f(x_1) + (x_3j^2 - x_1j^2)f(x_2) + (x_1j^2 - x_2j^2)f(x_3)}{(x_2j - x_3j)f(x_1) + (x_3j - x_1j)f(x_2) + (x_1j - x_2j)f(x_3)} \]#

For 3 parents see above.

This will find the parabola connecting 3 current points and produce a single child at min(x) of that parabola. 

\section{Design a suitable fitness function}

f(x)=y 

|f(x)-y| 

\section{Describe what evolutionary operators you would use}

\section{Describe the selection scheme you would use}

Tournament (\mu + \lambda) - selection. \mu parents generated by Fitness Proportional Selection with a scaled fitness function of exponetial scaling. This is because exponential scaling includes a temperature that aproaches 0 as time goes on.

Fitness Proportional Selection

\bold{Pr}[x] = \frac{(f(x)}{\sum_y\mem(P) f(y)}.
Exponential Scaling:

 f~(x) := exp(f(x)/T),	where T > 0

\section{Describe the experiments that you would carry out and the quantities that you would report to assess your solution. (You do not need to implement this but you need to describe your design.)}


\end{document}


